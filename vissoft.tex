\documentclass[conference]{IEEEtran}

\usepackage{graphicx}
\usepackage{hyperref}
\usepackage{xspace}
\usepackage{listings}
\usepackage[usenames, dvipsnames]{color}

\lstset{language=Python}

\lstdefinestyle{custompython}{
	belowcaptionskip=1\baselineskip,
	frame=lr,
	xleftmargin=\parindent,
	language=Python,
	basicstyle=\footnotesize\ttfamily,
	keywordstyle=\bfseries\color{MidnightBlue},
	stringstyle=\color{PineGreen},
  commentstyle=\color{Magenta}
}
         

\graphicspath{{./img/}}

\hyphenation{op-tical net-works semi-conduc-tor}


\newcommand{\tool}{Dashboard\xspace}

\usepackage{fourier-orns}

\definecolor{myblue}{RGB}{120, 131, 205}
\definecolor{myviolet}{RGB}{187, 101, 202}


\newcommand{\niceseparator}
	{
		\begin{center}
  		% $\ast$~$\ast$~$\ast$
  		% $\clubsuit$~$\clubsuit$~$\clubsuit$
  		\leafleft
		\end{center}
	}

\newcommand{\epDecoration}[1]{{\small {\bf #1}}\xspace}

\newcommand{\epTranslations}{\epDecoration{api.get\_possible\_translations}}
\newcommand{\epOutcome}{\epDecoration{api.report\_exercise\_outcome}}

\definecolor{mlcolor}{RGB}{140, 140, 205}
\newcommand{\ml}[1]{ 
	{\footnotesize \color{mlcolor}ML: #1}
	}

\newcommand{\mltp}[1]{\ml{Thijs, Patrick: #1}}
\newcommand{\mlv}[1]{\ml{Vasilios: #1}}





\begin{document}
%
\title{\tool: A Lightweight Analytics Platform for Visualizing Evolving Service Performance in Python }
% Alternative Titles: The Importance of Visualization in the Performance Monitoring of Python Web Services


% author names and affiliations
% use a multiple column layout for up to three different
% affiliations
%\author{\IEEEauthorblockN{Michael Shell}
%\IEEEauthorblockA{School of Electrical and\\Computer Engineering\\
%Georgia Institute of Technology\\
%Atlanta, Georgia 30332--0250\\
%Email: http://www.michaelshell.org/contact.html}
%\and
%\IEEEauthorblockN{Homer Simpson}
%\IEEEauthorblockA{Twentieth Century Fox\\
%Springfield, USA\\
%Email: homer@thesimpsons.com}
%\and
%\IEEEauthorblockN{James Kirk\\ and Montgomery Scott}
%\IEEEauthorblockA{Starfleet Academy\\
%San Francisco, California 96678--2391\\
%Telephone: (800) 555--1212\\
%Fax: (888) 555--1212}}

\author{
\IEEEauthorblockN{NAMES ORDER TBA}\\
Johann Bernoulli Institute of Mathematics and Computer Science\\
University of Groningen\\
Groningen, the Netherlands\\
Email: \{v.andrikopoulos,m.f.lungu\}@rug.nl, \{t.klooster.1,p.p.vogel\}@student.rug.nl
}

% make the title area
\maketitle

\begin{abstract}
The abstract goes here.
\end{abstract}

% no keywords

\IEEEpeerreviewmaketitle



\section{Introduction}
Every system is a distributed system nowadays \cite{cavage2013there}. 
% \hfill mds
 
% \hfill August 26, 2015

Python is currently one of the most popular programming languages. At the time of writing this paper\footnote{June 2017} Python is the 4th most popular programming language cf. the Tiobe Index\footnote{TIOBE programming community index is a measure of popularity of programming languages, created and maintained by the TIOBE Company based in Eindhoven, the Netherlands}. 

\todo{Flask summary goes here}

A search on GitHub with the keyword ``language:Python'' returns more than 500K open source projects written in the language. If we restrict the search by adding the keyword ``Flask'' we obtain a listing of 25K projects, that is, 5\% of all the Python projects. Flask -- advertised as a {\em micro-framework} -- is a lightweight alternative to web site and service development. 


However, there is no dedicated solution for monitoring the performance of Flask web-applications. Thus, every one of those Flask projects faces one of the following options when confronted with the need of gathering insight into the runtime behavior of their implemented service: 

  \begin{enumerate}

    \item Use a heavyweight professional API monitoring setup they usually require setting up a different server and can not take advantage of the already existing 

    \item Implement their own analytics tool 

    \item Live wihout analytics insight into their services \footnote{This is very real option: and is exactly what happened to the API that will be presented in this case study for many months. }

  \end{enumerate}

\todo{For the first point in the list, we can also argue that analytics solutions like Google Analytics can be used, but they have no notion of versioning/integration with the development lifecycle. Feel free to cite \cite{papazoglou2011managing} for service evolution purposes}

For projects which are done on a budget (e.g. research projects, startups) the first and the second options are often not available due to time and financial constraints. 

To avoid these projects ending up in the third situation, in this paper we present a low-effort, lightweight service monitoring API for Flask and Python web-services.

To start using our Python library for service visualization solution one needs to one line of to connect their Flask application object with the dashboard and one extra line of code to import the library:

% dashboard.config.from_file('dashboard.cfg')
\begin{lstlisting}[caption=Configuring the \tool is straightforward,style=custompython]

  import dashboard
  dashboard.bind(app=flask_app)

\end{lstlisting}


\mltp{a small description of how the dashboard automatically intercepts the calls to the various API calls}

\mltp{metion also that the dashboard automatically is available at /dashboard endpoint}

\mltp{a small screenshot of how the dashboard allows one to select the interesting }

\section{Case Study}

  \zee is a platform and an ecosystem of applications for accelerating vocabulary acquisition in a foreign language \cite{Lungu16} and is used as a case study for the remainder of this paper where all the figures are captured from the actual deployment of \tool in the context of the \zee platform\footnote{Within the \tool the figures are interactive offering basic data exploration capabilities: filter, zoom, and details on demand\cite{Shne99a}}

  The architecture of the ecosystem has at its core an API implemented with Flask and Python and a series of satellite applications that together offer three main intertwined features for the learner:

  \begin{enumerate}

    \item Reader applications that provide effortless translations for those texts which are too difficult for the readers

    \item Interactive exercises personally generated based on the preferences and past likes of the learner

    \item Article recommendations which are at the appropriate level of difficulty for the reader. The difficulty is estimated based on past exercise and reading activity

  \end{enumerate}

  The core API provides correspondingly three types of functionality: contextual translations, article recommendations, and personalized exercise suggestions. The core API of system is a research project, which sustains at the moment the reading and practice of about one hundred and fifty beta-testers. 

  \subsection* {Structure of the Paper}
  The remainder of this paper is structured as follows: In Sections \ref{sec:util}, \ref{sec:perf}, \ref{sec:user} we dedicate a separate section to three types of analysis that \tool supports and present a the way the visualizations support these types of analysis. 

\ml{we should consider adding also one section in which the architecture/implementation and main features of the dashboard are presented before going on with discussing them in more depth in the following sections --- this should include a rundown on which views are provided from where (overview or per endpoint)}



\section{Overall Endpoint Utilization}
\label{sec:util}

  The most fundamental insight that a service maintainer needs regards service utilization. \vspace{0.5cm}

  Figure \ref{fig:aeu} shows a first perspective on endpoint utilization that \tool provides: a stacked bar chart of the number of hits to various endpoints grouped by day \footnote{We recommend obtaining a color version of this paper for better readability} shows that at its peak the API has about 2500 hits per day. 
  The way users interact with the platform can also be inferred since the endpoints are indicators of different activity types, e.g.: 

  \begin{enumerate}

    \item {\color{myblue}\epTranslations} is an indicator of the amount of reading the users are doing

    \item {\color{myviolet} \epOutcome} is an indicator of the amount of vocabulary practice the users are doing

  \end{enumerate}


  \begin{figure}[h!]
    \centering
    \includegraphics[width=\linewidth]{all_endpoints_usage.png}
    \caption{The number of requests per endpoint per day view shows the overall utilization of the monitored application}
    \label{fig:aeu}
  \end{figure}

  This visualization also provides feedback to the maintainer when deciding about endpoint deprecation, the most elementary way of {\em understanding the needs of the downstream} \cite{Haen14a}. In our case study, the maintainer decided to not remove several endpoints once they saw that, contrary to their expectations, they were being used.\footnote{Usage information can also be used to increase the confidence of the maintainer that a given endpoint is not used, although it is not a proof.}

  \niceseparator

  A second type of {\em utilization} question that an API maintainer can answer by using the \tool regards cyclic patterns of usage during various times of day. 
  % \mlv{can we provide some support for this claim? a reference maybe?}

  % \mltp{ can we add vertical lines that highlight the beginng of a new week (e.g. before Sunday): }

    \begin{figure}[h!]
      \centering
      \includegraphics[width=\linewidth]{daily_patterns}
      \caption{Usage patterns become easy to spot in the requests per hour heatmap}
      \label{fig:dp}
    \end{figure}


  Figure \ref{fig:dp} shows the users of the Zeeguu API not practicing  languages at night, but otherwise hitting the API around the clock with several hundred hits per hour. 


\section{Visualizing Service Performance}
\label{sec:perf}

  The \tool also collects information regarding endpoint performance. The view in Figure \ref{fig:ep} summarizes the response times for various endpoints by using box-and-whisker plots. 

  % \ml{Thijs and Patrick... the visualization people will complain when they see that we have different colors for the same endpoint in different graphs. Can we insure that there is consistency in colors? Simplest trick would be to obtain the color by hashing the name of the endpoint... in that case the same endpoint woudl have the same color in various graphs.}

  \begin{figure}[h!]
    \centering
    \includegraphics[width=\linewidth]{endpoint_performance.png}
    \caption{The response time (in ms) per monitored endpoint view allows for identifying performance variability and balancing issues}
    \label{fig:ep}
  \end{figure}

  After investigating this view it became clear to the maintainer that three of the endpoints had very large variation in performance. One of the three was most critical was optimized first: the \epTranslations is part of a live interaction and it having such variable performance was a usability problem for the users of the reader applications. 

  % most important endpoints that they decided to improve the performance of is the second from the top in Figure \ref{fig:ep}: the endpoint that returns translations. This is particularily critical since it is part of a live interaction loop. 

\newpage
  \niceseparator

  To be able to see their improvements in action, the maintainer had to add an extra configuration information to be able to find the `.git' folder from where to retrieve the current version of the deployed application: 

    \begin{lstlisting}
      [caption=Configuring the \tool with the path to the .git folder enables the generation of evolutionary performance graphs, style=custompython]

    dashboard.config.git = 'path/to/git-root/of/app'

    \end{lstlisting}

  After redeploying the API, the dashboard can now automatically detect the current version of the project, and can group measurements by version. \tool can now generate the view in Figure \ref{fig:tee} where the performance of the give endpoint is tallied by version.

    \begin{figure}[h!]
      \centering
      \includegraphics[width=\linewidth]{translation_endpoint_evolution.png}
      \caption{Visualizing The Performance Evolution of the \epTranslations endpoint}
      \label{fig:tee}
    \end{figure}

  This way the maintainer could confirm that the performance of the translation endpoint improved: in the latest version (bottom-most box plot in Figure \ref{fig:tee}) the entire box plot moved to the left and there are fewer outliers.


  \niceseparator

  The \tool also collects {\bf extra information about outliers}: Python stack trace, CPU load, request parameters, etc. in order to allow the maintainer to investigate the causes of these exceptionally slow response times. 

  In order to address this, but not degrade the usual performance the \tool tracks for every endpoint a running average value. When it detects that a given request is an outlier with respect to this past average running value, it triggers the {\em outlier data collection routine} which stores all the previously listed extra information about the current execution environment. 


\section {User Centered Visualization}
\label{sec:user}

  For service endpoints which run computations in real time as they are called, there might be very different timings based on the different loads that are sent to the endpoint. 

  In our cases study, one of the slowest endpoints, and one with the higest variability is \epFeedItems: it retrieves a list of recommended articles for a given user. However, since a user can be subscribed to anything from one to three dozen article sources, and since the computation of the difficulty is personalized and it is slow, the variability in time among users is likely to be very large. 

  \tool provides a way of grouping information on a per user basis. However, to do this, the developer must specify the way in which a given API call can be associated with a given user. There are multiple ways, the simplest takes again advantage of the strengths of the Flask framework which offers a global request object which contains session information: 

  \begin{lstlisting}[float,caption=TBA,style=custompython]
    
    # app specific way of extracting the user
    # from a flask request object    
    def get_user_id(request):
      sid = int(request.args['session'])
      session = User.find_for_session(sid)
      return user_id

    # attaching the get_user_id function
    dashboard.config.get_group_by = get_user_id

  \end{lstlisting}


  Sometimes, grouping the service calls per endpoint it is not sufficient. Figure \ref{fig:tpu} shows some of the results of calling the \epFeedItems endpoint for various users. 

  \begin{figure}[h!]
    \centering
    \includegraphics[width=\linewidth]{time_per_user}
    \caption{The \epFeedItems shows a very high variability across users}
    \label{fig:tpu}
  \end{figure}


  Different users might have different experiences: 
  - a user has 10K emails one has 10 emails
  - a user is subscribed to 20 feeds one to 2 feeds

  The system will have different processing times. 
  It is important for the DevOps-er to be able to understand the difference in performance on a per user basis. 

  If we try to show also per version: 

  \begin{figure}[h!]
    \centering
    \includegraphics[width=\linewidth]{time_per_user_per_version}
    \caption{Caption here}
    \label{fig:figure1}
  \end{figure}


\section{Discussion}

  \subsection{Integrating with Different Deployment Strategies in Order to Automatically Monitor System Evolution}

  The main goal of the \tool design was to allow analytics to be collected and insight to be gleaned by making the smallest possible changes to the running API. To allow the collection of evolutionary information 

  This technique assumes that the web server code which is the target of the monitoring is deployed using \git in the following way: 

  \begin{enumerate}
    \item The deployment engineer pulls the latest version of the code from the integration server; this will result in a new commit being pointed at by the HEAD pointer than previously
    \item The deployment engineer restarts the new version of the service. At this point, the \tool detects that a new HEAD is present in the local code base and consequently starts associating all the new data points with this new commit
  \end{enumerate}

  The advantage of this approach is the need for minimal configurability. The disadvantage is that it will consider the smallest of the commits\footnote{Even one that modifies a comment} and the shortest lived commits\footnote{A commit which was active only for a half an hour before a new version with a bug fix was deployed}, as a distinct way of grouping the data points. 

  Another possible extension point here is supporting other version control systems (e.g. Mercurial). However, this is a straightforward extension.


  \subsection{Tool and Source Code Availability}

    \tool is implemented for Python 3.6 and is available on the Python Package Index repository\footnote{\url{http://pypi.org/TODO}} from where it can be installed on any system that has Python installed by running \install from the command line. 

    The code of \tool is published under a permissive MIT license and is available on GitHub.\footnote{\url{https://github.com/mircealungu/automatic-monitoring-dasboard}}

    The images in this paper are screenshots of the interactive visualizations from the deployment of \tool in the context of the Zeeguu core API. The actual deployment can be consulted online by the reviewers and readers of this article\footnote{\url{https://zeeguu.unibe.ch/api/dashboard}. Username: {\em guest}, password: {\em vissoft}}.


\section{Related Work}

Java Visualization \cite{Pauw02a}

Run-time monitoring of services \cite{ghezzi2007run}

\section{Conclusion}
The conclusion goes here.




% conference papers do not normally have an appendix


% use section* for acknowledgment
\section*{Acknowledgment}


The authors would like to thank...





% references section

\bibliographystyle{IEEEtran}
\bibliography{vissoft}


% that's all folks
\end{document}


